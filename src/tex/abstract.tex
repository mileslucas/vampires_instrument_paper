\begin{abstract}
    stet
    % We present upgrades to the Visible Aperture-Masking Polarimetric Imager/Interferometer for Resolving Exoplanetary Signatures (VAMPIRES), a sub-instrument of the Subaru Coronagraphic Extreme Adaptive Optics (SCExAO) which enables diffraction-limited visible light imaging, reaching $\sim$\SI{20}{\mas} FWHM and 60\% Strehl ratios on-sky. We deployed two new qCMOS detectors which have \SIrange{0.22}{0.4}{\electron} RMS read noise and reach \SI{500}{\hertz} framerates for a \ang{;;3}x\ang{;;3} FOV. We developed a new multiband imaging technique using dichroics to multiplex four broadband images for spectral differential imaging and spectro-polarimetry. We have characterized all the filter zero points and various throughputs for flux calibration and the instrument astrometric solution. We characterize the coronagraphic optics and present on-sky contrast curves with classic Lyot coronagraphs reaching a 5$\sigma$ contrast of $10^{-4}$ at \ang{;;0.1}, $10^{-5}$ at \ang{;;0.2}, and $10^{-6}$ at $>$\ang{;;0.6} separations. We include the detection of the low-mass star HD 1160B with multiband coronagraphic observations. We describe the instrument polarization control and differential imaging procedure and show on-sky results of spectro-polarimetric imaging with the multiband imaging technique on the circumstellar disk host HD 169142. We also show high-resolution narrowband H$\alpha$ imaging of the symbiotic star R Aqr, showcasing a new asymmetry in the jet source emanating from the compact accretion disk. In culmination, we show that VAMPIRES is one of the most technically capable visible high-contrast instruments, currently, and has a promising future with the significant AO3k upgrades at Subaru.
\end{abstract}