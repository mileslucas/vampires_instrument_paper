\section{Photon number resolving statistics}\label{sec:pnr_derivation}

We derive the theoretical noise effects of photon-number resolving data \textit{post-hoc}, wherein data that is assumed to be perfectly bias-subtracted and flat-corrected is converted to electron flux using the detector gain and rounded to the nearest whole number.

Photons that enter the detector have a certain probability of becoming photo-electrons (hereafter electrons), described by the quantum efficiency. We ignore this for now and describe the input flux in electrons. These electrons are subject to counting statistics due to their Poisson nature. Let $H$ be the rate of electrons per pixel (the quanta exposure). The probability of measuring a signal of $k$ electrons is then
\begin{equation}
    \mathcal{P}\left(k | H\right) = \frac{H^k e^{-H}}{k!}.
\end{equation}

Now consider the Gaussian read noise, $\sigma_{RN}$
\begin{equation}
    \mathcal{N}\left(x | k, \sigma_RN\right) = \frac{1}{\sigma_{RN}\sqrt{2\pi}}e^{-(x - k)^2 / 2\sigma_{RN}^2}.
\end{equation}
The probability of measuring $U$ electrons after both processes is the convolution of the two distributions
\begin{equation}
    P(U | H, \sigma_{RN}) = \sum_{k=0}^\infty{\mathcal{N}\left(U | k, \sigma_{RN}\right)\cdot \mathcal{P}\left(k|H\right)}.
\end{equation}

Qualitatively, recognize that this can be described as discrete Poisson peaks convolved with a Gaussian of width $\sigma_{RN}$. Consider the simple case of $H=0$; the data is a Gaussian centered at zero. If this data is rounded to the nearest whole number, values $|U|>0.5$ get rounded to -1 and 1, respectively, or for $|U| > 1.5$ to -2 and 2, and so on, causing variance in the measured photon number.

Consider a new distribution describing the probability of measuring values which will be rounded to the correct bins, $h(k)$
\begin{align}
    h(k) = &P\left(|U - k| < 0.5 | \sigma_{RN}\right) \\
    &= \int_{k - 0.5}^{k + 0.5}{\mathcal{N}(x | k, \sigma_{RN}) dx}.
\end{align}
This can be rewritten using the cumulative distribution of the unit normal, $\Phi$, 
\begin{equation}
    h(k) = \Phi\left(\frac{k + 0.5}{\sigma_{RN}} \right) - \Phi\left(\frac{k - 0.5}{\sigma_{RN}} \right)
\end{equation}
with
\begin{equation}
    \Phi(z) = \frac12 \left[1 - \mathrm{erf}\left(\frac{z}{\sqrt{2}}\right) \right],
\end{equation}
where $\mathrm{erf}$ is the error function.

The expected value of this distribution is 0 by convenience of symmetry
\begin{equation}
    \mathrm{E}[h] = \sum_{k=-\infty}^{\infty}{ k \cdot h(k)} = 0,
\end{equation}
and the variance of this distribution is
\begin{equation}
\label{eqn:pnr_var}
    \mathrm{Var}[h] = \sum_{k=-\infty}^{\infty}{ k^2\cdot h(k)}.
\end{equation}
For practical applications, $k$ is truncated to \num{\pm5} since the probability of observing a Gaussian event over five standard deviations away is extremely small, below $10^{-5}$, and thus $h(k)\approx0$.

Therefore, the total noise from photon number resolving is the combination of photon noise and \autoref{eqn:pnr_var}--
\begin{equation}
    \label{eqn:pnr_std}
    \sigma = \sqrt{U + \sum_{k=-\infty}^{\infty}{ k^2\cdot \left[\Phi\left(\frac{k + 0.5}{\sigma_{RN}} \right) - \Phi\left(\frac{k - 0.5}{\sigma_{RN}}\right)\right]}}.
\end{equation}


In \autoref{fig:pnr}, the relative improvement in S/N by photon number resolving (i.e., converting the signal to electrons and rounding to the nearest whole number) is shown. We also show the standard deviation of Monte Carlo samples ($N=10^4$) with and without rounding to confirm our theoretical results. Note that there are more sophisticated algorithms for achieving higher S/N ratios with photon-number resolving than simply rounding after the fact \citep[see][]{harpsoe_bayesian_2012}, but this is still a somewhat contrived use case for a ground-based instrument like VAMPIRES, which will be photon-noise limited for most observations.

\begin{figure}
    \centering
    \script{pnr.py}
    \includegraphics[width=\columnwidth]{figures/pnr.pdf}
    \caption{Relative S/N improvement from calculating the photon number via truncation in the low-flux regime compared to the standard noise terms. Solid curves are theoretical values (\autoref{eqn:pnr_std}) and the scatter points are Monte Carlo statistical simulations shown for both the ``slow'' ($\sigma_{RN}$=\SI{0.25}{\electron}) and the ``fast''  ($\sigma_{RN}$=\SI{0.4}{\electron})  detector readout modes.\label{fig:pnr}}
\end{figure}
