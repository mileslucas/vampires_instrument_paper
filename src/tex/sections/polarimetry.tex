\section{Polarimetry}\label{sec:polarimetry}

VAMPIRES is capable of polarimetry through the use of a broadband half-wave plate (HWP) and a wire-grid polarizing beamsplitter cube. This puts orthogonal polarization states on each detector which can be subtracted to measure linear polarization \citep{kuhn_imaging_2001}. To measure the two linear Stokes quantities ($Q$ and $U$) we modulate the HWP at angles \ang{0}, \ang{45}, \ang{22.5}, \ang{67.5}, allowing double-differencing to remove a significant portion of instrumental polarization.

For observations with DIT $<$ \SI{1}{\second} a fast achromatic ferro-electric liquid crystal (FLC) device can be inserted into the beam and synchronously modulated with every exposure. In this mode we can more effectively subtract the signals from fast-moving speckles by triple-differencing with the FLC modulation \citep{norris_vampires_2015}. VAMPIRES also uses two quarter-wave plates (QWP) for correcting the static instrumental birefringence, in particular from the periscope in SCExAO.

Polarimetric measurements are limited by instrumental polarization due to the many inclined surfaces and reflections in the common path of VAMPIRES. It is a well-established technique to model and remove the instrumental effects using Mueller calculus \citep{holstein_polarimetric_2020,joost_t_hart_full_2021}. For VAMPIRES we use internal calibrations with a halogen flat lamp and linear polarizer after the telescope M3 as a polarized source. We modulate the facility HWP and image rotator and create sum and difference images from both detectors in order to fit our Mueller matrix model for each VAMPIRES filter.  These calibrations are set up to be highly-automated using the python control code. 

\subsection{Polarimetric Spectral Differential Imaging}

We have developed a new mode for VAMPIRES for polarimetric measurements in the narrowband filters. Because the narrowband filters are stored in the differential filter wheel only one camera receives light from the filter. To enable polarimetry we must switch the differential filter between the cameras for each HWP position so that we measure both horizontal and vertically polarized light for each filter. In post-processing one can choose to do standard double- or triple-differencing separately for each filter to get the Stokes data, or the two filters can be subtracted immediately before differencing for spectral differential imaging (SDI).

Polarimetric SDI (PSDI) will allow direct measurement of the polarized fraction of narrowband emission, which is particularly interesting for H$\alpha$ imaging of circumstellar disks. For example, the H$\alpha$ emission around AB Aur may be from a forming protoplanet \citep{currie_images_2022}, in which case the emission should be unpolarized. If the H$\alpha$ emission is polarized this points to different hypothesis, such as stellar emission scattering off dust grains \citep{zhou_uv-optical_2023}. This technique has been tested with an internal calibration source and future observations will be used to demonstrate our method on-sky.