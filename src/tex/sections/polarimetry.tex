\section{Imaging polarimetry}\label{sec:polarimetry}

VAMPIRES is capable of polarimetric differential imaging (PDI; \citet{kuhn_imaging_2001}), which is a powerful technique for high-contrast observations. Because VAMPIRES measures orthogonal polarization states simultaneously through nearly the same optical path, any unpolarized signal can be canceled to a precision of TODO, including the variable PSF speckles. This differential technique significantly suppresses the stellar PSF to reveal faint polarized circumstellar signal, like from planet-forming disks or exoplanets. Through polarimetric modulation, VAMPIRES can remove or greatly reduce the effects of instrumental polarization and fast atmospheric seeing in polarimetric measurements.

\subsection{Polarimetric differential imaging}

VAMPIRES is designed to measure linear polarization, which is described by the Stokes parameters $Q$ and $U$. We measure the linear polarization via a wire-grid polarizing beamsplitter cube which will transmit horizontally polarized light ($I_{0^\circ}$) to VCAM1 and vertically polarized light ($I_{90^\circ}$) to VCAM2. This is equivalent to using a gray beamsplitter with a horizontal/vertical linear polarizer in front of each camera. In the reference of the VAMPIRES instrument bench, the Stokes Q value can be  measured with
\begin{equation}
    Q(x, y) = I_{0^\circ}(x, y) - I_{90^\circ}(x,y)
\end{equation}
To measure Stokes $U$ we take the difference of light at a \ang{45} angle from horizontal,
\begin{equation}
    U(x, y) = I_{45^\circ}(x, y) - I_{135^\circ}(x,y)
\end{equation}
This simple subtraction will remove the majority of unpolarized signal, leaving only instrumental effects. Furthermore, there is only one set of filters in between the beamsplitter cube and the cameras, so the non-common path aberrations between cameras are minimal. However, because it is two separate cameras with unique detector characteristics it is critical to have accurate frame calibration and alignment of signal frame camera to camera.

Because we use a polarizing beamsplitter we cannot reorient the analyzer state to measure $U$, so instead we use a half-wave plate (HWP) to retard our analyzer state by the appropriate angle.
\begin{align}
    X_\theta(x, y) = I_{2\theta}(x, y) - I_{2\theta + 90^\circ}(x, y)
\end{align}
Therefore, the Stokes parameters can be observed through sequential images at the following HWP angles
\begin{align}
\begin{split}
    Q^+(x, y) &= X_{0^\circ}(x, y) \\
    Q^-(x, y) &= X_{45^\circ}(x, y) \\
    U^+(x, y) &= X_{22.5^\circ}(x, y) \\
    U^-(x, y) &= X_{67.5^\circ}(x, y)
\end{split}
\end{align}
A second layer of differential processing occurs through this waveplate switching, where
\begin{align}
\begin{split}
    \label{eqn:doublediff}
    Q(x, y) &= 0.5\cdot\left[Q^+(x, y) - Q^-(x, y)\right] \\
    U(x, y) &= 0.5\cdot\left[U^+(x, y) - U^-(x, y)\right]
\end{split}
\end{align}
This modulation will effectively cancel any instrumental noise downstream of the waveplate, and therefore it is imperative to place it as far up the optical path as possible. The Subaru waveplate unit (WPU, \citealt{watanabe_near-infrared_2018}) is mounted after M3 before the AO188 optics (\autoref{fig:schematic}), which is as far upstream as feasibly possible considering the size of the telescope M1-M3 mirrors.

For the total intensity, we take the sums of frames rather than the differences
\begin{align}
\begin{split}
    I_Q(x, y) &= I_{0^\circ}(x, y) + I_{90^\circ}(x,y) \\
    I_U(x, y) &= I_{45^\circ}(x, y) + I_{135^\circ}(x,y)
\end{split}
\end{align}
In theory $I_Q$ and $I_U$ are equivalent, but instrumental polarization and non-simultaneity will lead to slight differences. We can estimate the Stokes $I$ intensity from the average of $I_Q$ and $I_U$. From the resulting $I$, $Q$, and $U$ values for each pixel we assemble a Stokes cube where each pixel is the vector quantity
\begin{equation}
    \mathbf{S}(x, y)= \left[\begin{matrix}
        I(x, y) & Q(x, y) & U(x, y)
    \end{matrix}\right]
\end{equation}

\subsection{High-speed polarimetric modulation}

VAMPIRES is equipped with a fast FLC modulator as another layer of differential polarimetric control. The principal of fast modulation is the same as the HWP modulation explained above- when the FLC is active the optic acts like a HWP oriented at \ang{45} which switches the input state into an orthogonal state. By switching orthogonal polarization states at the same rate or faster than then atmospheric speckle timescale (\si{\milli\second}) we can differentially remove these fast speckles instead of creating a halo when averaging frames together. This fast modulation technique has been proven successfully on VAMPIRES \citep{norris_vampires_2015}, SPHERE/ZIMPOL \citep{schmid_spherezimpol_2018}, and other instruments (TODO ARE THERE OTHERS?).

We upgraded VAMPIRES with a new achromatic FLC (design wavelength of \SI{700}{\nano\meter}) which improves the non-ideal retardance at the shortest and longest wavelengths in VAMPIRES. Additionally, studies of the previous FLC showed alarming results, such as a retardance much closer to a QWP than a HWP, that may be due to FLC aging. Our FLC is able to modulate with a switching time $<$\SI{100}{\micro\second} at room temperature, which is sufficiently fast enough to synchronize with every exposure of our detectors without significant duty cycle losses. The FLC and both cameras components are all synchronized electronically with a \SI{120}{\mega\hertz} micro-controller. Because the SCExAO bench is often at an ambient temperature of \SIrange{0}{20}{\celsius} we custom modified the FLC enclosure to add a resistive heating element which is controlled to keep the FLC optical tube at \SI{45}{\celsius}, ensuring no thermal effects will slow this switching speed down.

Lastly, we had to incorporate new, larger mounting optics and added a motion stage to allow removing the FLC from the beam. Our FLC is restricted to modulation periods $<$\SI{1}{\second}. For exposures longer than \SI{1}{\second}, if we leave the FLC switching during an exposure the light will cancel out since we image orthogonal polarization states on the same detector- we will have effectively created an optical shutter. We cannot leave the FLC unmodulated, though, since without any current across the crystal the FLC will have a randomized polarization state. So to enable long-exposure polarimetric observations with HWP modulation, only, we now can remove the FLC.

\subsection{Correcting instrument birefringence}

One complicated aspect of the SCExAO layout is the optical periscope directing light from the SCExAO infrared bench out of plane up into the visible-light bench.  The periscope introduces a significant amount of pupil rotation and birefringence in the instrument. VAMPIRES uses two QWPs mounted before the first periscope mirror to correct for a majority of this birefringence. The pair of QWPs are tuned to an optimal angle pair such that 100\% horizontally polarized light entering the WPU polarizer is 100\% vertically polarized entering the VAMPIRES polarizing beamsplitter. The reason for the change in orientation is for polarimetric considerations of common-path instruments, like FIRST.

These QWPs have high utility, though, and can be used as another level of slow modulation for calibration or polarimetric control. Future work will explore how to improve the polarimetric efficiency and precision through tracking laws which can correct, for example, the non-ideal polarizing effects of the image rotator \citep{joost_t_hart_full_2021,zhang_characterizing_2023}. We also consider future investigations into how to optimize the instrument polarization for the DGVVC, which uses a polarization grating for diffraction control (\autoref{sec:coronagraphy}).

\subsection{Instrumental modeling and correction}

Polarimetric measurements are limited by instrumental polarization due to the many inclined surfaces and reflections in the common path of VAMPIRES, in particular from M3. It is a well-established technique to model and remove the instrumental effects using Mueller calculus \citep{holstein_polarimetric_2020,joost_t_hart_full_2021}. The Stokes parameters measured in \autoref{eqn:doublediff} through differencing are a measurement of the input Stokes values modified by the telescope and instrument Mueller matrix
\begin{equation}
    \mathbf{S}(x, y) = \mathbf{M}\cdot\mathbf{S}_{in}(\alpha, \delta)
\end{equation}
where $\mathbf{S}$ represents stokes vector at each point in the field, either detector or sky coordinates. By calibrating the Mueller-matrix, $\mathbf{M}$, we can invert this equation to retrieve the Stokes vectors unmodified by instrumental effects. For VAMPIRES we use internal calibrations with a halogen flat lamp and linear polarizer after the telescope M3 as a polarized source for internal calibrations \citep{zhang_characterizing_2023}. We modulate the facility HWP and image rotator and create sum and difference images from both detectors in order to fit our Mueller-matrix model for each VAMPIRES filter. These calibrations are set up to be highly-automated to enable consistent monitoring of the polarimetric performance of VAMPIRES.

As mentioned above, M3 is a significant source of polarimetric noise due to the small F-ratio ($f$/14) and \ang{45} AOI. This causes splitting of the s- and p-polarization states in a variety of polarimetric aberrations called the Goos-Hanchen and Imbert-Federov shifts and tilts \citep{schmid_spherezimpol_2018,van_holstein_polarization-dependent_2023}. Because our internal polarized source is after M3, these birefringent effects cannot be easily calibrated without on-sky observations. Measurements of unpolarized ($p<$0.01\%) and polarized ($p>$1\%) standard stars in order to calibrate the effects of M3 and determine absolute polarimetric precision, and a full characterization is future work (PI: Zhang, M.). We recommend observers to plan observations of an unpolarized standard star for any high-precision polarimetry: at least one HWP cycle at the beginning and end of your science sequence.

\subsection{Polarimetric Spectral Differential Imaging}

We have developed a new mode for VAMPIRES for polarimetric measurements in the narrowband filters, which had previously not been explored despite being technically possible. Because the narrowband filters are stored in the differential filter wheel only one camera receives light from any specific narrowband filter. To enable polarimetry we must switch the differential filter between the cameras at each HWP position so that we measure both horizontal and vertically polarized light for each filter (four total raw data cubes per HWP position). In post-processing one can choose to do standard double- or triple-differencing separately for each filter to get the Stokes data, or the two filters can be subtracted immediately before differencing for polarimetric spectral differential imaging (PSDI). If SDI is not desired and exposure times are fast enough, only one camera can be used with the FLC for double-difference PDI in a single narrowband filter.

Polarimetric SDI (PSDI) will allow direct measurement of the polarized fraction of narrowband emission, which is particularly interesting for H$\alpha$ imaging of circumstellar disks. For example, the H$\alpha$ emission around AB Aur may be from a forming protoplanet \citep{currie_images_2022}, in which case the emission should be unpolarized. If the H$\alpha$ image is polarized, though, this points to a different hypothesis, such as stellar emission scattering off dust grains \citep{zhou_uv-optical_2023}. By using SDI the H$\alpha$ image should have the majority of any stellar continuum signal removed, giving a cleaner observable for understanding these complex stellar environments. This technique has been tested with an internal calibration source and future observations will be used to demonstrate our method on-sky.