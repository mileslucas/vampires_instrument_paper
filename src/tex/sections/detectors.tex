\section{Detectors}\label{sec:detectors}

For our upgrades we deployed two custom-modified \textit{Hamamatsu ORCA-Quest} qCMOS detectors\footnote{\url{https://www.hamamatsu.com/us/en/product/cameras/qcmos-cameras/C15550-20UP.html}}. These detectors were chosen for their fast readout ($\sim$500 Hz at 512x512), low read noise ($\sim$ 0.2 - 0.4 $e^-$ RMS), and high dynamic range ($\sim$ 85 - 90 \si{\decibel}). We modified the cameras to be compatible with the telescope liquid cooling loop which keeps the detectors at \SIrange{-45}{-40}{\celsius} without creating turbulence or vibrations in the instrument from air cooling.

The detectors have two  readout modes: the full-speed ``FAST'' mode, and the slower, but more sensitive, ``SLOW'' mode. In ``FAST'' mode there is an electronic shutter that allows detector integration times (DIT) of \SI{7.2}{\micro\second} - \SI{1800}{\second}. In ``SLOW'' mode the DIT is limited to $\gtrsim$\SI{50}{\milli\second}, depending on the camera crop. In this mode the analog to digital converters benefit from the reduced readout rate to reach $\sim$0.2 $e^-$ RMS read noise.

The detectors have rolling-shutter readout that exposes four pixel rows at a time. The maximum framerate is limited by the number of rows read out and the readout speed described previously. In addition, the rolling-shutter effect becomes more prevalent at the shortest DITs and the duty cycle gets lower and lower. At the extreme, each group of four pixel rows will not be exposed until after the preceding group has been read out. As an additional complication, when using the hardware trigger the detectors have reduced framerates and duty cycle, below 50\% in some scenarios, to enable consistent timings for synchronization between both cameras.

\subsection{Detector characterization}

We measured a photon transfer curve using a calibration flat field source for both detectors in both readout modes using a non-polarizing beamsplitter cube. We removed the fixed-pattern noise through frame-differencing before performing a linear fit between the average signal and the signal variance \citep{stefanov_cmos_2022}. 
\begin{equation}
    \sigma^2\left(\mu | k, \sigma_{RN}\right) = \frac{\mu}{k} + \sigma^2_{RN}
\end{equation}
From this fit we extract the gain ($k$) in \si{\electron/adu} from the slope of the line. We measure the readout noise from the standard deviation of a bias frame and convert to \si{\electron} with the gain. We observe that the full well (\SI{7000}{\electron}) is greater than the maximum bit-depth with our measured detector gain. The detector photoresponse is linear over the entire dynamic range and saturates at $2^{16}$ \si{adu}. During commissioning we also took dark frames with the detector caps covered. For data with significant signal, we performed a linear fit of the signal to the exposure time to get the average dark current per pixel. We report our results in \autoref{tbl:detectors}.

We compare the sensitivity of the qCMOS detectors to the EMCCD detectors in \autoref{fig:detector_snr_relative}. These signal-to-noise (S/N) curves use calibrated read noise, dark current, and gain along with manufacturer quantum efficiency (QE) values averaged over \SIrange{600}{800}{nm}. For the EMCCDs we include a clock-induced charge (CIC) flux of \SI{1e-3}{\electron/pixel/frame} based on empirical tests with our detectors. When electron multiplication is enabled we add a excess noise factor of $\sqrt{2}$ \citep{harpsoe_bayesian_2012}. The qCMOS detectors have an order of magnitude worse dark current and slightly worse QE compared to the EMCCDs. Despite this, \autoref{fig:detector_snr_relative} shows that the qCMOS detectors have better performance than the EMCCDs when photon-noise or read-noise limited, which is essentially all standard use-cases except for long exposures. The qCMOS detectors are dark-limited in ``SLOW'' mode at $\sim$\SI{15}{s}, where the relative performance in the low-flux regime drops compared to the EMCCDs. Using the EMCCDs without electron multiplication is superior at high photon fluxes due to the lack of excess noise factor (compared to the EMCCD) combined with higher QE (compared to the qCMOS).

\begin{figure}
    \centering
    \script{detector_snr.py}
    \includegraphics[width=\columnwidth]{figures/detector_snr_relative.pdf}
    \caption{Theoretical normalized S/N curves for the new CMOS detectors (red curves) and the previous EMCCD detectors (blue curves). The curves are normalized to an ideal camera (shot-noise only with perfect QE). The top plot shows a read noise-limited case, and the bottom plot shows a dark-limited case. Noise terms include read noise, dark noise, photon noise, and CIC plus excess noise factor for the EMCCDs.\label{fig:detector_snr_relative}}
\end{figure}

The qCMOS detectors are capable of photon-number resolving- that is, statistically determining the number of electrons from Poisson statistics. The photon number statistic is essentially free from frame-to-frame noise (like readout noise), but requires accurate bias subtraction and flat-fielding for practical use. We demonstrate the benefit of photon number resolving in \autoref{fig:pnr}. The top plot shows a histogram of pixel values in a 100 by 100 pixel window for a long exposure dark frame. The photon peaks from Poisson statistics are clear. We use the method from \citet{starkey_determining_2016} to determine the gain and quanta exposure directly from the histogram peaks. 

In the bottom plot is the relative improvement in S/N by photon number resolving (i.e., converting signal to electrons and rounding to nearest whole number). The derivation of our formula is in \autoref{sec:pnr_derivation}. We also show the standard deviation of Monte Carlo samples ($N=10^4$) with and without rounding to confirm our theoretical results. We note that there are more sophisticated algorithms for achieving higher S/N ratios with photon-number resolving than simply rounding after the fact \citep{harpsoe_bayesian_2012}, but this is still a somewhat contrived use case for a ground-based instrument like VAMPIRES which will be photon-noise limited for most observations.

\begin{figure}
    \centering
    \script{pnr.py}
    \includegraphics[width=\columnwidth]{figures/pch.pdf}
    \caption{Histogram of pixel values in a 100 by 100 pixel region of a long exposure in ``SLOW'' mode ($\sigma_{RN}$=\SI{0.22}{\electron})- all signal is due to dark current. The Poisson photon number peaks are clearly resolved.\label{fig:pch}}
\end{figure}

\begin{figure}
    \centering
    \script{pnr.py}
    \includegraphics[width=\columnwidth]{figures/pnr.pdf}
    \caption{Relative S/N improvement from calculating the photon number in the low-flux regime compared to the standard noise terms. Solid curves are theoretical values (\autoref{eqn:pnr_std}) and the scatter points are Monte Carlo statistical simulations shown for both the ``SLOW'' and the ``FAST''  ($\sigma_{RN}$=\SI{0.4}{\electron}) detector readout modes.\label{fig:pnr}}
\end{figure}
Overall, our detectors are some of the most sensitive available commercially today. Compared to electron-multiplication technologies, the calibration and operation of the detectors is much simpler. In addition, the previous installment of VAMPIRES did not allow exposure times $>$\SI{1}{s}, so our new detector architecture is vastly more flexible and sensitive than before. We recommend flat-fielding to correct for the pixel-to-pixel gain variations that are much higher in CMOS detectors. The bias frames are quite structured due to the CMOS architecture, so we also recommend bias or sky-background frames for subtracting the structured offset. Other than that, the detectors are virtually free of cosmetics so we only apply bad-pixel correction to long exposures for cosmic-ray rejection.

\begin{deluxetable}{cccccc}
\tablehead{
    \multirow{2}{*}{Cam} & 
    \multirow{2}{*}{Mode} & 
    \colhead{Gain} & 
    \colhead{RN} & 
    \colhead{DC} & 
    \colhead{DR} \\
     &
     &
    \colhead{(\si{\electron/\adu})} &
    \colhead{(\si{\electron})} &
    \colhead{(\si{\electron/\second/\pixel})} &
    \colhead{(\si{\decibel})}
}
\tablecaption{VAMPIRES detector characteristics.\label{tbl:detectors}}
\startdata
\multirow{2}{*}{VCAM1} & FAST & 0.103 & 0.40 & \num{3.6e-3} & 85 \\
 & SLOW & 0.105 & 0.25 & \num{3.6e-3} & 90 \\
\multirow{2}{*}{VCAM2} & FAST & 0.103 & 0.40 & \num{3.6e-3} & 85 \\
 & SLOW & 0.105 & 0.22 & \num{3.6e-3} & 90 \\
\enddata
\tablecomments{The input-referred dynamic range (DR) is derived from max signal ($2^{16}$ \si{\adu} $\approx$ \SI{6800}{\electron}) divided by the readout noise.}
\end{deluxetable}


% \begin{figure}
%     \centering
%     \script{detector_snr.py}
%     \includegraphics[width=\columnwidth]{figures/detector_snr.pdf}
%     \caption{Theoretical S/N curves for the new CMOS detectors (red curves) and the previous EMCCD detectors (blue curves). The top plot shows a read noise-limited case, and the bottom plot shows a dark-limited case. Noise terms include read noise, dark noise, photon noise, and excess noise factor for the EMCCDs.\label{fig:detector_snr}}
% \end{figure}

\subsection{Astrometric calibration}

The astrometric calibration is used to determine the instrument rotation angle and pixel plate scale. We determine our calibration solution using visual binaries and correlate those solutions to the pinhole mask in SCExAO. This way future observers can use the pinhole mask calibration data taken off-sky as an absolute astrometric reference. 

For our observations we observed three visual binaries (Albireo, 21 Oph, and Algol), but we only use one set of data due to systematic errors in two of the data sets. During our observations of Albireo there was significant optical vignetting due to a misplaced wheel in the pupil plane and our observations of Algol are very defocused. Thankfully we have multiple nights of data from 21 Oph in all filters, including the multiband mode. We also attempted to use a globular cluster (M5), but the lack of good guide stars for the adaptive optics system made our images too seeing limited and too low S/N for accurate astrometry.

\begin{deluxetable}{ccccl}
\tablehead{
    \multirow{2}{*}{Object} &
    \colhead{$t_\mathrm{obs}$} &
    \colhead{$\rho$} &
    \colhead{$\theta$} &
    \multirow{2}{*}{Ref.} \\
    &
    \colhead{(MJD)} &
    \colhead{($^{\prime\prime}$)} &
    \colhead{(\textdegree)} &
}
\tablecaption{Visual binaries used for astrometric calibration. The ephemeris (separation, $\rho$, and position angle East of North, $\theta$) are reported for the average observation time of the data.\label{tbl:binaries}}
\startdata
\multirow{2}{*}{21 Oph} & 60132.26 & \num{0.8\pm0.039} & \num{-53.0\pm3.1} & \multirow{2}{*}{[1]} \\
 & 60135.24 & \num{0.8\pm0.039} & \num{-53.0\pm3.1} & \\
\enddata
\tablerefs{[1]: \cite{docobo_new_2007,docobo_iau_2017}}
\end{deluxetable}

We reduced our images of 21 Oph using typical dark subtraction and derotated the images by their parallactic angle. We then mean-combined the derotated images and calculated the phase auto-correlation, \textit{a la} speckle interferometry, forward-propagating the statistical uncertainties. We used a parabolic fit to the peak in the corellelogram created by the secondary component to determine the relative offset from the primary component. We bootstrap resampled the phase corellelograms \num{100} times to determine the uncertainty in these centroids.
From there we computed the separation and position angle of the secondary and compared to the ephemerides computed using published orbital elements \citep{docobo_new_2007,docobo_iau_2017}. This gives us the detector plate scale in \si{mas/px} and the parallactic angle offset in degrees, which are shown in \autoref{tbl:astrometry}.

We describe the instrument angle as the pupil rotation of the instrument with respect to the AO188 optical bench, where the image rotator zero point is calibrated. For on-sky observations, this is calculated by
\begin{equation}
    \theta_\mathrm{inst} = \theta_\mathrm{off} + 180\text{\textdegree} - \theta_\mathrm{PAP}
\end{equation}
where $\theta_\mathrm{PAP}$ is the static image rotator pupil offset of \ang{-39} (to align with the SCExAO entrance pupil). We note a \ang{180} parity compared to the previous astrometric solution for the EMCCDs \citep{currie_images_2022}.

\begin{deluxetable}{cccc}
\tablehead{
    \multirow{2}{*}{Cam} &
    \colhead{px. scale} &
    \colhead{$\theta_\mathrm{off}$} &
    \colhead{$\theta_\mathrm{inst}$} \\
    &
    \colhead{(mas/px)} &
    \colhead{(\textdegree)} &
    \colhead{(\textdegree)}
}
\tablecaption{Astrometric characteristics of VAMPIRES from observations of 21 Oph on 2023 \formatdate{7}{7}{2023} and \formatdate{10}{7}{2023}.\label{tbl:astrometry}}
\startdata
VCAM1 & \num{6.03\pm0.29} & \num{100.4\pm3.1} & \num{-40.6\pm3.1} \\
VCAM2 & \num{6.04\pm0.29} & \num{99.6\pm3.1} & \num{-41.4\pm3.1} \\
\enddata
\end{deluxetable}

We separately characterized each camera and the use of the multiband imaging dichroics (MBI), although we did not see significant differences with or without the MBI dichroics. Pinhole mask data were obtained off-sky on the nights of our observations and were dark-subtracted and median-combined. Using the plate scale and parallactic angle offset fit previously we determine the relative spacing and rotation of the pinholes. These results are shown in \autoref{tbl:astrometry}. Our astrometric results are completely dominated by the statistical uncertainty in the binary orbital elements. Further analysis of binaries with high-quality orbital solutions or globular clusters with space-based data (e.g., from Hubble Space Telescope) are required for improved astrometry and are left for future work.
