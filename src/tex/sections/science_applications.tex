\section{Science applications}\label{sec:science}

\subsection{Exoplanets in reflected light}

\subsection{Polarimetric imaging of circumstellar disks}

\subsection{Stellar mass-loss shells and atmospheres}

\subsection{Close binary stars}

Stellar companions in the form of binary or multiple star systems are well-suited for orbital characterization and photometric analysis with VAMPIRES. The high-angular resolution achievable with lucky imaging ($\approx$\SIrange{20}{30}{mas}; \autoref{fig:binary}) is only possible with visible-light instruments on \SI{10}{\meter}-class telescopes. The high-contrast achieved through wavefront correction with AO188 and SCExAO also expands the magnitude limits of companions to fainter targets. With multiband imaging the relative position of the companion at each wavelength allows \textit{spectro-astrometry} WHICH IS COOL??TODO??.

Examples of unique science applications at visible wavelengths are hot, faint companions, such as H$\alpha$ components to M-giants (like in R Aqr; \autoref{sec:raqr}, \citet{schmid_spherezimpol_2017}), or white dwarf companions of Barium stars (TODO CITATION).

\subsection{Solar-system objects}

As demonstrated in \autoref{sec:neptune} VAMPIRES is capable of imaging solar-system bodies such as planets, bright asteroids, the Galilean moons, etc. The high angular resolution and polarimetric capabilities are useful for measuring sizes, shapes, and mapping surface features in high detail. The multiband imaging capabilities of VAMPIRES allows spectropolarimetric analysis of atmospheric or terrestrial scattering. The main limitations for solar-system are the loss of wavefront control efficiency from using extended objects as the natural guide stars and the \ang{;;3} FOV of the instrument.

