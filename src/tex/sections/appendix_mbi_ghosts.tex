\section{Multiband Imaging Ghosts}\label{sec:ghosts}

When stacking dichroics with low angles of incidence, extra reflections create ghost PSFs. Our multiband dichroic technique (\autoref{sec:mbi}) uses three dichroics in front of a mirror with $\sim$\ang{0.4} tilts between each filter with respect to the mirror. We characterized the ghost PSFs using a dispersing prism to understand their origin by analyzing spectra.
\begin{figure}
    \centering
    \script{mbi_ghosts_diagram.py}
    \includegraphics[width=0.85\columnwidth]{figures/mbi_ghost_raytrace.pdf}
    \caption{Raytrace diagram for ghosts from stacked dichroics. D1 represents a dichroic tilted at angle $\theta$, and M represents a standard mirror. The incoming ray (red arrow) goes from left to right. The light reflected by D1 is shown with a green arrow. Light transmitted by D1 is blue-- the solid line is the main ray and the dashed line is the first ghost ray.\label{fig:mbi_ghost}}
\end{figure}

To explain, consider the simple case of a single dichroic with a tilt, $\theta$, in front of a mirror (\autoref{fig:mbi_ghost}). There are three rays to explain the ghost behavior-- the first is the light reflected by the dichroic (green arrow), which has an angle of $-2\theta$. The second ray is transmitted through the dichroic, reflects off the mirror, and returns with an angle of \ang{0} (solid blue arrow). Because the dichroic has parallel faces, there is no angular deviation of the transmitted rays. The angle between the solid blue and green arrows ($2\theta$) is the characteristic separation between the fields in the focal plane.

The final ray is formed from a transmitted beam that reflects off the rear mirror and then off the dichroic's backside (dashed blue arrow). The reflection is a natural consequence of the dichroic filter curve-- any light that is not fully transmitted will be reflected by the dielectric coating and return towards the mirror, picking up an angle of $2\theta$ due to the tilt of the dichroic. This beam reflects off the rear mirror again and then is transmitted through the dichroic with a ray angle of $2\theta$. This will form a PSF with the same spectrum as ``PSF 2'' but separated by an angle of $2\theta$. More back reflections can occur every time the light transmits through the dichroic, reflecting a small fraction of the light at integer multiple separations of $2\theta$ (i.e., $4\theta$, $6\theta$, and so on). These higher-order ghosts are less concerning because they are exponentially fainter and far from the primary PSFs.

One way to remove these ghosts would be to improve the dielectric coatings of the filters until the ghosts are undetectable. Our multiband dichroics have a transmission of $\sim$95\% outside of the OD=$4$ blocking region, meaning $\sim$5\% of light is reflected into these ghosts and easily detected. It is difficult and expensive to manufacture dichroics that simultaneously have efficient blocking (OD$>$3) in band and high throughput out of band ($>$99.9\%), so we opted to address the ghosts geometrically-- creating a focal plane layout where no ghosts overlap with any science fields.

Our solution uses optimal angles for the dichroic tilts which place any ghost PSFs (and their \ang{;;3} FOV) completely out of the FOV of the primary fields (\autoref{fig:mbi_fields}). Our design uses as few pixel rows as possible to maximize detector readout speed. Additionally, a single field can be cropped out to double the readout speed leaving three fields remaining. For the final design, angles were chosen for each dichroic slightly larger than \ang{;;3} so that the manufacturing tolerances would allow for at least $\sim$\SI{10}{\pixel} between fields.

\begin{figure}
    \centering
    \script{mbi_field_diagram.py}
    \includegraphics[width=\columnwidth]{figures/mbi_field_diagram.pdf}
    \caption{Diagram of the fields produced by the VAMPIRES multiband dichroics. Each field is colored based on the expected spectrum, from brown (F760) to orange (F610). Each field is labeled in the bottom left with the angular displacement ($x,y$) of the field in terms of $\varphi=2\theta$, the magnified angle offset between dichroic filters in the MBI stack.\label{fig:mbi_fields}}
\end{figure}