\section{The VAMPIRES point-spread function}\label{sec:psf}

One of the key parameters of a high-contrast imaging instrument is the instrumental point-spread function (PSF) and how close the on-sky performance reaches the ideal PSF. We are able to simulate the ideal PSF using our knowledge of the instrument filter bandpasses, a detailed description of the telescope pupil, and the appropriate plate scales for the focal plane. We use \texttt{HCIpy} \citep{por_high_2018} for PSF simulations using straightforward Fraunhofer diffraction and an analytical function describing the Subaru/SCExAO pupil. This simulation code open-source and available to anyone interested in simulating the SCExAO or VAMPIRES PSF.

The observed VAMPIRES point spread function (PSF) can vary wildly depending on the residual wavefront error and the exposure time. Because the PSF fluctuates rapidly from residual atmospheric wavefront errors, exposures with DIT longer than the coherence time ($\sim$\SI{10}{\milli\second}) will average together many speckle realizations, smoothing out the instrumental PSF. With fast exposure times, though, the PSF can be ``frozen'', which allows discarding frames with low PSF quality and removing a large portion of the wavefront error encapsulated in tip and tilt by co-aligning frames. This is commonly referred to as ``lucky imaging'' or ``shift-and-add'' \citep{garrel_highly_2012}.

We show an annotated PSF in the F720 filter of \textit{HD 191195} after post-processing in \autoref{fig:onsky_psf}. This PSF is the combination of \num{3e4} individual \SI{5.1}{\milli\second} exposures (about \SI{2}{\minute} total). The data was dark-subtracted and lucky-imaged by discarding the worst 75\% of frames and aligning with cross-correlation with a synthetic PSF. The collapsed frames were combined after the fact with a weighted mean. This data has a Gaussian FWHM of \SI{22}{\mas} and a Strehl\footnote{We measure Strehl using a broadband Synthetic PSF normalized to the photometric flux in an aperture of \ang{;;1}.  We subsample the peak with a Fourier transform, as in method seven (7) of \citep{jr_is_2004}.} of 70\%, which is close to the values we see with our internal laser source.

\begin{figure}
    \centering
    \script{onsky_psfs.py}
    \includegraphics[width=\columnwidth]{figures/onsky_psfs.pdf}
    \caption{On-sky images of \textit{HD 183143} in the F720 filter (DIT=\SI{5.1}{\milli\second}). Data has been aligned and collapsed and is shown on a log scale with clipping to emphasize the fainter regions of the image. (G1-G4) passive speckles created by the diffractive gridding of the SCExAO deformable mirror. (S1-S2) Spider diffraction spikes. (CR) the SCExAO control radius where a speckle halo creates an apparent dark hole. (QS) quasi-static speckles created by residual and non-common path wavefront errors.\label{fig:onsky_psf}}
\end{figure}

In this image wee many typical features of high S/N VAMPIRES observations. We refer to the annotations in \autoref{fig:onsky_psf} and describe them here. First, the center of the frame contains the classical obstructed Airy disk pattern with a bright central core and first ring. Wavefront errors perturb the primary Airy ring in these observations, which otherwise should be a radially symmetric ring. Beyond the first Airy ring is a plateau of signal from the Subaru secondary mirror obstruction. (QS) Beyond this plateau are the quasi-static speckles which are due to residual wavefront errors which the adaptive optics system cannot sense or correct.

The PSF is not azimuthally symmetric with two large, bright diffraction features from the secondary mirror support structure, called the spider diffraction spikes (S1-S2). These spikes can greatly reduce achievable contrast if they align in the field with a target, but using coronagraphy (\autoref{sec:coronagraphy}) or exploiting field rotation can mitigate the effects of the diffraction spikes.

The DM of SCExAO has $\sim$45 actuators across the pupil, which limits the lowest spatial frequency corrected by the AO loop to $\sim$22.5$\lambda/D$. Typically this is referred to as the control radius or ``dark hole'', and it is clearly visible in our PSF image (CR). The dark hole is closer to a square in this image, which is a function of the modal basis used for wavefront sensing and control. The DM also acts like a diffractive element due to its actuator grid which creates copies of the stellar PSF at four locations orthogonal to the orientation of the DM (G1-G4). These passive speckles should have a fixed astrometric and photometric relation to the star but suffer from chromatic aberration and appear smeared at their far separations from the star (especially in the Open filter). In very high S/N data, like from the internal laser, lower order copies of the PSF appear at multiples of 45$\lambda/D$.

\subsection{Seeing and low-wind effect}
In \autoref{fig:bad_psfs} we show less-than-optimal PSF images which have poor wavefront correction or suffer from the notorious low-wind effect. The first of these images is of a faint star ($m^R=$8.1) with 380 x \SI{0.5}{\second} exposures reduced in a similar manner as above, except without any frame-selection and using a Gaussian model for frame-registration. The PSF is completely seeing-limited with a Gaussian FWHM of \ang{;;0.1} and estimated Strehl of 14\%. 

The low-wind effect (LWE) is the realization of wavefront discontinuities across the telescope pupil spiders  \citep{cantalloube_origin_2018}. Physically this is attributed to a local thermodynamic disequilibrium from the M2 support spiders, which have absorb and transfer heat differently than the air around them. This causes ``petaling'' in the sectors of the telescope pupil which splits the PSF into a clover pattern. LWE is especially detrimental because it is largely unsensed by current wavefront sensors. We observe some LWE at some point in the night for almost every VAMPIRES observation, but it depends on atmospheric conditions and can be mitigated to an extent through frame-selection. We show a single \SI{}{\milli\second} frame with notable LWE.


\begin{figure}
    \centering
    \script{bad_psfs.py}
    \includegraphics[width=\columnwidth]{figures/bad_psf_mosaic.pdf}
    \caption{.\label{fig:bad_psfs}}
\end{figure}

\subsection{Radial profiles}
We plot the radial profile of our high-quality and seeing-limited PSFs along with a synthetic instrument PSF in \autoref{fig:onsky_psf_profiles}. The profiles are normalized to each other using relative aperture photometry with a  diameter of \ang{;;1}. All the profiles are normalized so the ideal PSF has a peak value of 1. We also show the cumulative aperture sums, or encircled energy, normalized to a value of 1 at \ang{;;0.5}.
\begin{figure*}
    \centering
    \script{onsky_psf_profiles.py}
    \includegraphics[width=\textwidth]{figures/onsky_psf_profiles.pdf}
    \caption{\label{fig:onsky_psf_profiles}}
\end{figure*}

We see that our high-quality PSF closely matches the profile of the ideal PSF until about 15$\lambda/D$. The nulls of the radial pattern are not as deep in our data as the model due to quasi-static speckles. Past 15$\lambda/D$ the control radius of the SCExAO DM is clear due to the residual atmospheric speckle halo. The overall dynamic range from peak to apparent noise floor is \SI{90}{\decibel}.