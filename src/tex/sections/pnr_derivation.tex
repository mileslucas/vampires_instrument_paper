\section{Photon number resolving statistics}\label{sec:pnr_derivation}
Let $g(x)$ a zero-centered Gaussian of std $\sigma$

$$
g \sim \mathcal{N}\left(0, \sigma_{RN}\right)
$$

Rounding operation turns $g$ into a discrete distribution $h$. I'm writing this as a function but really it's a sequence of Diracs over $\mathbb{Z}$. Like, a discrete distribution (\emph{insert more sass here.}).
\begin{align}
h(0) &= \int_{-0.5}^{0.5}g(x) dx\\
h(1) = h(-1) &= \int_{0.5}^{1.5}g(x) dx\\
h(k) = h(-k) &= \int_{k - 0.5}^{k + 0.5}g(x) dx\\
\end{align}

Now by convenience symmetry, the mean of the distro $h$ is 0.
\begin{equation}
\mathrm{E}[h] = \sum_{-\infty}^{+\infty} h(k) \cdot k = 0
\end{equation}
The variance is:
\begin{equation}
\mathrm{V}[h] = \sum_{-\infty}^{+\infty} h(k) \cdot k^2 = 2 \cdot \sum_{1}^{+\infty} h(k) \cdot k^2
\end{equation}
And now since $g$ is a Gaussian we can rewrite $h$ in terms of erf differences, and good to go we are.
\begin{equation}
h(k) = h(-k) = \int_{k - 0.5}^{k + 0.5}g(x) dx
\end{equation}
\begin{equation}
    h(\pm k) = \mathrm{erf}\left(\dfrac{k+0.5}{\sigma_{RN}}\right) - \mathrm{erf}\left(\dfrac{k-0.5}{\sigma_{RN}}\right)
\end{equation}

\begin{equation}
\label{eqn:var_pnr}
\mathrm{V}[h] = 2 \cdot \sum_{k=1}^{+\infty}  k^2 \left[%
\mathrm{erf}\left(\dfrac{k+0.5}{\sigma_{RN}}\right) - \mathrm{erf}\left(\dfrac{k-0.5}{\sigma_{RN}}\right)%
\right]
\end{equation}

We verify the solution in \autoref{eqn:var_pnr} via Monte Carlo simulation for $\sigma_{RN}\in\{$0.22, 0.4$\}$\si{\electron} (\autoref{tbl:detectors}). We draw $N=$\num{100000} samples from a Poisson distribution and then add $N$ samples from a normal distribution with zero mean and $\sigma_{RN}$ standard deviation
\begin{equation}
    s \sim \mathcal{N}\left(\mathcal{P}\left(f\right), \sigma_{RN}\right)
\end{equation}
For the standard case we calculate the standard deviation of all the samples for each flux value, $f$. For the photon number resolving case we round $s$ to the nearest integer and then calculate the standard deviation. \autoref{fig:pnr_improvement} shows the relative improvement of the S/N beetween the photon number resolving case and the standard case (both theoretical and Monte Carlo results). From \autoref{fig:pnr} we expect a deterioration in S/N from rounding to to photon numbers for $\sigma_{RN}<$\SI{-1}{\electron}, which is reflected by the worse performance in the ``Fast'' readout mode.


\begin{figure}
    \centering
    \script{pnr.py}
    \includegraphics[width=\columnwidth]{figures/pnr_improvement.pdf}
    \caption{.\label{fig:pnr}}
\end{figure}

