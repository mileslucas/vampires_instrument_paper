\section{Photon number resolving statistics}\label{sec:pnr_derivation}

We derive the theoretical noise effects of photon-number resolving data \textit{post-hoc}, wherein data that is assumed to be perfectly bias-subtracted and flat-corrected is converted to electron flux using the detector gain and rounded to the nearest whole number.

Photons which enter the detector have a certain probability of becoming photo-electrons (hereafter electrons) described by the quantum efficiency. We ignore this for now and describe our input flux in electrons. These electrons are subject to counting statistics due to their Poisson nature. Let $H$ be the number of electrons (or the rate of electrons) in one pixel (the quanta exposure). The probability of measuring a signal of $k$ electrons is then
\begin{equation}
    Y \sim \mathcal{P}(H)
\end{equation}
\begin{equation}
    P(Y=k | H) = \frac{H^k e^{-H}}{k!}
\end{equation}

This variable then encounters Gaussian read noise, $\sigma_{RN}$, which is described by the convolution of a Poisson and Gaussian distribution
\begin{equation}
    X \sim \mathcal{N}(Y, \sigma_{RN})
\end{equation}
The probability of measuring $U$ electrons is then
\begin{equation}
    P(X = U | H, \sigma_{RN}) = \sum_{k=0}^\infty{\frac{1}{\sigma_{RN}\sqrt{2\pi}}e^{-(U - k)^2 / 2\sigma_{RN}^2}\cdot P(Y=k|H)}
\end{equation}
Qualitatively, we recognize that this can be described as discrete Poisson peaks which get widened by a Gaussian of width $\sigma_{RN}$. Consider the simple case of $H=0$, the data is simply a Gaussian centered at zero. If we take this data and round it to the nearest whole number, there is a small chance for values $|U|>0.5$  to get rounded to -1 and 1, respectively, causing variance in the measured photon number. We can describe the probability of measuring values which will be rounded to the correct bins.

\begin{equation}
    P\left(|U - k| < 0.5 | \sigma_{RN}\right) = \int_{k - 0.5}^{k + 0.5}{\mathcal{N}(x | k, \sigma_{RN}) dx}
\end{equation}

This can be rewritten using the cumulative distribution of the unit normal, $\Phi$
\begin{equation}
    \Phi(z) = \frac12 \left[1 - \mathrm{erf}\left(\frac{z}{\sqrt{2}}\right) \right]
\end{equation}
where $\mathrm{erf}$ is the error function.
\begin{equation}
    P(|U - k| < 0.5 | \sigma_rn) = \Phi\left(\frac{k + 0.5}{\sigma_{RN}} \right) - \Phi\left(\frac{k - 0.5}{\sigma_{RN}} \right)
\end{equation}
The variance of this distribution is
\begin{equation}
\label{eqn:pnr_var}
\mathrm{Var} = \sum_{k=-\infty}^{\infty}{ k^2\cdot \left[\Phi\left(\frac{k + 0.5}{\sigma_{RN}} \right) - \Phi\left(\frac{k - 0.5}{\sigma_{RN}}\right)\right]}
\end{equation}
Therefore the total noise from photon number resolving is the photon noise plus the square root of \autoref{eqn:pnr_var}.
\begin{equation}
    \label{eqn:pnr_std}
    \sigma = \sqrt{U + \sum_{k=-\infty}^{\infty}{ k^2\cdot \left[\Phi\left(\frac{k + 0.5}{\sigma_{RN}} \right) - \Phi\left(\frac{k - 0.5}{\sigma_{RN}}\right)\right]}}
\end{equation}
