\section{Introduction}\label{sec:intro}

The advent of \SI{8}{\meter}-class telescopes combined with the advances in extreme adaptive optics has revolutionized astronomy by enabling high-angular resolution and high-sensitivity imaging and spectroscopy \citep{guyon_extreme_2018,currie_direct_2023,galicher_imaging_2024}. By employing cutting-edge technologies, innovative observing techniques, and sophisticated post-processing methods, high-contrast imaging (HCI) mitigates the diffraction effects of the stellar point spread function (PSF), enabling the study of faint circumstellar signals.

In recent years, the majority of HCI science has occurred at near-infrared (NIR) and thermal-infrared (TIR) wavelengths (\SIrange{1}{5}{\micron}) because longer wavelengths are affected less by atmospheric turbulence \citep{fried_optical_1966,roddier_effects_1981}. Currently, HCI instruments can reach $>$90\% Strehl ratios, a measure of the PSF quality, in good conditions in the NIR \citep{beuzit_sphere_2019,lozi_status_2020}. Stable, high-quality PSFs are required for efficient diffraction control via coronagraphic or interferometric techniques.

Visible-light high-contrast imaging instruments are more sensitive to optical path differences than the NIR \citep{fried_optical_1966,roddier_effects_1981}. Even the most advanced AO systems can only reach 40\%-60\% Strehl ratios at visible wavelengths \citep{ahn_scexao_2021,males_magao-x_2022}. This can be mitigated with ``lucky imaging'', where images are taken faster than the speckle coherence time ($\tau_0 \sim$\SI{4}{\milli\second}, \citealp{kooten_climate_2022}) ``freezing'' the speckle pattern, allowing individual frames with high-quality point spread functions (PSFs) to be aligned and combined while discarding poor-quality frames \citep{law_lucky_2006, garrel_highly_2012, lang_tractor_2016}.

Despite the challenges in wavefront control, visible-light instruments have some key advantages \citep{close_into_2014}.  Visible wavelengths have a smaller diffraction limit and, therefore, higher angular resolution than the NIR. The sky background is orders of magnitude fainter in the visible. Stellar emission lines in the visible (e.g., H$\alpha$, OIII, SII) have much higher contrasts compared to their local continuum than in the NIR (e.g., Pa$\beta$). Visible detectors have better noise characteristics and operational flexibility than infrared detectors, including more compact assemblies, simpler cooling requirements, and lower cost.

Specific science cases for visible HCI include circumstellar disks \citep{benisty_optical_2023}, exoplanet and sub-stellar companions \citep{hunziker_refplanets_2020}, evolved stars' atmospheres and mass-loss shells \citep{norris_vampires_2015}, stellar jets \citep{schmid_spherezimpol_2017,uyama_monitoring_2022}, accreting protoplanets \citep{uyama_high-contrast_2020,benisty_circumplanetary_2021}, close binaries \citep{mcclure_binary_1980,escorza_barium_2019}, and solar system objects \citep{schmid_limb_2006,vernazza_vltsphere_2021}.

Visible-light high contrast instruments include SCExAO/VAMPIRES \citep{norris_vampires_2015}, SPHERE/ZIMPOL \citep{schmid_spherezimpol_2018}, MagAO-X/VisAO \citep{males_magao-x_2024}, and LBT/SHARK-VIS \citep{mattioli_shark-vis_2018}. These instruments share some commonalities. They are all deployed on large telescopes with extreme AO correction-- high-order deformable mirrors and high framerate, photon-counting wavefront sensors. VisAO, SHARK-VIS, and VAMPIRES use high-framerate detectors for lucky imaging. All instruments have some form of narrowband spectral emission line imaging. Finally, VAMPIRES and ZIMPOL offer polarimetric modes, which use high-speed polarimetric modulation to combat atmospheric seeing.

This work presents significant upgrades to VAMPIRES, improving its capabilities as a high-contrast imager. These upgrades revolved around the deployment of two CMOS detectors with high-framerate photon-number-resolving capabilities, a suite of coronagraphs optimized for the visible, a new multiband imaging mode for simultaneous imaging in multiple filters, and a new achromatic liquid crystal polarization modulator. These upgrades significantly improve VAMPIRES' high-contrast capabilities, reaching deeper sensitivity limits with improved efficiency.
