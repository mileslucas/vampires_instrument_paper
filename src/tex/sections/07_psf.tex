\section{The VAMPIRES point-spread function}\label{sec:psf}

One of the key parameters of a high-contrast imaging instrument is the instrumental point-spread function (PSF) and how close the on-sky performance reaches the ideal PSF. The observed VAMPIRES PSF quality varies depending on the residual wavefront error and the exposure time. Because the PSF fluctuates rapidly from residual atmospheric wavefront errors, integrating longer than the speckle coherence time ($\sim$\SI{4}{\milli\second}, \citealp{kooten_climate_2022}) blurs the PSF. The speckled diffraction pattern can be ``frozen'' with fast exposure times, enabling lucky imaging. The high-speed detectors in VAMPIRES can easily take exposures faster than the coherence timescale, which makes it well-suited for this technique.


An annotated PSF in the F720 filter of HD 191195 after post-processing is shown in \autoref{fig:onsky_psf}. This data has a FWHM of \SI{20}{\mas} and a Strehl ratio\footnote{The Strehl ratio is measured using a broadband synthetic PSF normalized to the photometric flux in an \ang{;;0.5} radius aperture. The peak is subsampled with a Fourier transform, as in method seven (7) of \citet{roberts_is_2004}.} of $\sim$52\%.

\begin{figure}
    \centering
    \script{onsky_psfs.py}
    \includegraphics[width=\columnwidth]{figures/onsky_psfs.pdf}
    \caption{On-sky images of \textit{HD 191195} in the F720 filter (DIT=\SI{2.3}{\milli\second}). Data has been aligned and collapsed and is shown on a log scale with clipping to emphasize the fainter regions of the image. (G1-G4) passive speckles created by the diffractive gridding of the SCExAO deformable mirror. (S1-S2) Spider diffraction spikes. (CR) the SCExAO control radius, where a speckle halo creates an apparent dark hole. (QS) quasi-static speckles created by residual and non-common path wavefront errors.\label{fig:onsky_psf}}
\end{figure}

\subsection{PSF Features}
The PSF image in \autoref{fig:onsky_psf} shows typical features of high S/N VAMPIRES observations. The center of the frame contains the central core and the first few Airy rings, which are slightly asymmetric due to wavefront errors. Beyond this are the quasi-static speckles (QS), which slowly evolve due to non-common path errors \citep{soummer_speckle_2007,ndiaye_calibration_2013}.

The PSF contains two large, bright diffraction spikes from the secondary mirror support structure (spiders; S1-S2). These spikes reduce image contrast if they align with a target in the field, but using coronagraphy (\autoref{sec:coronagraphy}) or exploiting field rotation over time can mitigate the effects of these spikes.

The SCExAO deformable mirror has 45 actuators across the pupil, which limits the highest spatial frequency corrected by the AO loop to $\sim$22.5$\lambda/D$. Typically, this is referred to as the control radius or ``dark hole,'' visible in the PSF image (CR). The dark hole is closer to a square in this image, determined by the modal basis used for wavefront sensing and control. The DM also acts like a diffractive element due to the gridding of the surface, which creates copies of the stellar PSF at four locations orthogonal to the orientation of the DM (G1-G4).

\subsection{Seeing and the Low-Wind Effect}
The low-wind effect (LWE) is caused by piston wavefront discontinuities across the telescope pupil spiders  \citep{milli_low_2018}. The effect in the focal plane is ``splitting'' of the PSF (\autoref{fig:bad_psfs}). Current wavefront sensors struggle to detect LWE, which periodically occurs in almost every VAMPIRES observation, although it can be mitigated to an extent through frame selection. The other PSF image is of a faint star ($m_{\rm R}=$8.1) with \SI{0.5}{\second} exposures. The PSF is fuzzy with a FWHM of \SI{75}{\mas} and low S/N resulting from the blurring of the residual speckle pattern throughout the integration.

\begin{figure}
    \centering
    \script{bad_psfs.py}
    \includegraphics[width=\columnwidth]{figures/bad_psf_mosaic.pdf}
    \caption{(Left) High-framerate image in good conditions shows a high-quality PSF (Strehl $\sim$52\%). (Middle) high-framerate image highlighting LWE. (Right) a longer exposure image in poor seeing conditions is blurred into a Moffat-style profile. All images are shown with a square-root stretch with separate limits.\label{fig:bad_psfs}}
\end{figure}

\subsection{Lucky Imaging}

\autoref{fig:lucky_imaging} shows two observing scenarios: high framerate data (\SI{500}{\hertz}) in median seeing conditions and low framerate data (\SI{2}{\hertz}) in poor seeing conditions. A long exposure was simulated by taking the mean image; then, the data was registered using centroids. Progressively, from left to right, more frames are discarded based on the Strehl ratio, which shows that frame selection increases the PSF quality, but in poor conditions, it lowers the overall S/N for little gain.

\begin{figure*}
    \centering
    \script{lucky_imaging.py}
    \includegraphics[width=\textwidth]{figures/lucky_imaging_mosaic.pdf}
    \caption{Post-processing data with lucky imaging. Each frame is shown with a log stretch and separate limits. (Top) high-framerate (\SI{500}{\hertz}) data in median seeing conditions. (Bottom) low-framerate (\SI{2}{\hertz}) data in mediocre seeing conditions. (Long Exp.) a mean combination without alignment, simulating a long exposure. (Shift-and-add) co-registering each frame before collapsing. (Discarding X\%) same as shift-and-add but discarding a percentage of data based on the Strehl ratio.\label{fig:lucky_imaging}}
\end{figure*}

\subsection{Radial Profiles}
The radial profiles of a high-quality PSF, a seeing-limited PSF, and a synthetic instrument PSF are shown in \autoref{fig:onsky_psf_profiles}. The ideal PSF is normalized to a peak value of 1 and the other profiles are normalized to the ideal PSF using a \ang{;;1} aperture sum. The cumulative aperture sums, or encircled energy, are normalized to a value of 1 at \ang{;;1} and shown alongside the radial profiles. The good, on-sky PSF closely matches the profile of the theoretical PSF until about 15$\lambda/D$. The nulls of the radial pattern are not as deep as the model due to quasi-static speckles. Past 15$\lambda/D$, the control radius of the SCExAO DM is clear due to the residual atmospheric speckle halo.

\begin{figure*}
    \centering
    \script{psf_profiles.py}
    \includegraphics[width=\textwidth]{figures/psf_profiles.pdf}
    \caption{Radial profiles and encircled energy in different observing scenarios. (Left) radial profiles normalized to an ideal PSF (gray), a high-quality on-sky PSF (the same as \autoref{fig:onsky_psf}, solid red), a poor-quality on-sky PSF (the same as \autoref{fig:bad_psfs}, dashed red), and an on-sky coronagraphic PSF with the \SI{59}{\mas} IWA (solid blue). The $1\sigma$ raw contrast curves are shown for the good PSF (dotted red) and the coronagraphic PSF (dotted blue). The control radius of the SCExAO DM is marked with a vertical gray line. (Right) the encircled energy normalized to an ideal PSF with a max radius of \ang{;;1} (gray), a good on-sky PSF (solid red), and a poor-quality PSF (dashed red). The control radius of the SCExAO DM is marked with a vertical gray line.\label{fig:onsky_psf_profiles}}
\end{figure*}

