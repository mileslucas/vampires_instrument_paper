\section{Coronagraphy}\label{sec:coronagraphy}

As a high-contrast testbed SCExAO seeks to push wavefront control and coronagraphy to its limits. Currently VAMPIRES is equipped with four Lyot-style coronagraphs with various focal plane mask sizes and a new vector-vortex phase mask. The focal plane masks are mounted in a precise three-axis translation stage for fine alignment due to the lack of a dedicated tip-tilt mirror for VAMPIRES. The four Lyot-style focal plane masks were designed to be partially transmissive (design 0.1\%; actual 0.6\%) circular dots with inner working angles (IWA) corresponding to roughly two, three, five, and seven resolution elements (2, 3, 5, and 7 $\lambda/D$) in radius. The new double-grating vector vortex coronagraph (DGVVC; PI: Doelman, D.) was deployed in November 2023. This mask requires further testing before being made available for observers, especially given the polarizing effects of the liquid crystal grating in the mask.

\subsection{Focal-plane masks}

A new set of Lyot focal plane masks were made for VAMPIRES for our upgrades so that each mask has a \SI{3}{\arcsecond}x\SI{3}{\arcsecond} fieldstop. These masks were constructed the same as \citet{lucas_visible-light_2022}, however we note the measured IWAs have changed slightly compared to the previous coronagraph masks. Flat-field images of the four CLC masks are shown in \autoref{fig:clc_masks} and for the DGVVC in \autoref{fig:dgvvc_masks}. The DGVVC field shows a faint azimuthal structure when using the polarizing beamsplitter due to the polarization effects of the mask itself. These features mostly disappear when using the non-polarizing beamsplitter.

\begin{figure}
    \centering
    \script{clc_masks.py}
    \includegraphics{figures/clc_masks.pdf}
    \caption{Flat-field images of the classic Lyot coronagraph (CLC) focal plane masks cropped to the inner \ang{;;2} FOV. Dust particles on the front and rear faces of the optic blemish the field.\label{fig:clc_masks}}
\end{figure}

\begin{figure}
    \centering
    \script{dgvvc_masks.py}
    \includegraphics{figures/dgvvc_masks.pdf}
    \caption{Flat-field images of the double-grating vector vortex coronagraph (DGVVC) cropped to the inner \ang{;;2} FOV. The left image is with the polarizing beamsplitter (PBS)  which shows an azimuthal variation in intensity. (Right) with the non-polarizing beamsplitter (NPBS) the effect is reduced, but not completely. The limits have been clipped to emphasize the azimuthal pattern. The frames show the amplitude mask obscuring the central defect of the mask.\label{fig:dgvvc_masks}}
\end{figure}

The focal plane masks all suffer to some degree from dust features on the front and back surfaces. These features  are problematic due to the transmission losses localized around dust particles. Flat-fielding can mitigate these effects but is hard to do successfully with VAMPIRES because the coronagraph mask mount is moved to align to the star, rather than the other way around. A dedicated tip-tilt solution for VAMPIRES before the coronagraph would be preferred, but space constraints and restrictions due to the common path optics of the PyWFS make this practically difficult (\autoref{fig:cad}).

We measured the mask IWAs by rastering the SCExAO internal calibration source across the focal plane mask and measuring the photometric throughput. We convert the motion stage position to on-sky angle using plate scale of \SI{1.8}{\arcsecond\per\milli\meter} (Lozi~J. private communication). We measured the flux as the total sum of the full frame. We normalize the throughput to the peak flux of the PSF far from the mask and measure the IWA where this normalized throughput is 50\%. We plot the throughput curves and IWA in \autoref{fig:iwa} and summarize the results in \autoref{tbl:coronagraph}.

\begin{deluxetable}{lcc}
\tablehead{\colhead{Name} & \colhead{Radius (\si{\micron})} & \colhead{IWA (\si{\mas})}}
\tablecaption{VAMPIRES coronagraph mask specifications.\label{tbl:coronagraph}}
\startdata
CLC-2 & 46 & 37 \\
CLC-3 & 69 & 59 \\
CLC-5 & 116 & 105 \\
CLC-7 & 148 & 150 \\
DGVVC & 7 & 61 \\
\enddata
\tablecomments{The radius of the DGVVC is the radius of the amplitude mask obscuring the central defect in the phase mask.}
\end{deluxetable}

\begin{figure}
    \centering
    \script{iwa.py}
    \includegraphics{figures/coronagraph_iwa.pdf}
    \caption{Normalized off-axis throughput for the various coronagraph masks in VAMPIRES measured on \formatdate{17}{11}{2023}. The throughputs are normalized so the minimum flux is 0 and the maximum flux is 1. The inner working angle (IWA) is marked with vertical dotted lines at the point where the throughput reaches 50\%.\label{fig:iwa}}
\end{figure}

\subsection{Pupil masks}

VAMPIRES has three Lyot stop masks for rejecting the light diffracted by the coronagraphic focal plane mask. There is one Lyot stop with moderate throughput (66\%) which is described in \citet{lucas_visible-light_2022}. This mask is coated with gold for high reflectivity, aiding alignment with the pupil camera and for a future low-order wavefront sensor. Two new masks were laser-cut from metal foil sheets and deployed in January 2024 with alternate geometry in order to improve the throughput of the masks (PI: Walk, A.). We show them alongside the VAMPIRES pupil in \autoref{fig:pupil_masks}.

We measure the geometric throughput $T_\mathrm{geom}$ of each mask by finding the ratio of pupil flux with and without the mask. We used the pupil imaging lens in the Open filter for our analysis and took images with each mask. We dark subtract and then median collapse the pupil images and then binarize the data to 0 or 1 with a threshold to remove non-uniform pupil illumination from the fiber-fed calibration source. We measure the geometric throughput with the ratio of the sums of each binarized pupil image to the unobstructed pupil.

We also measure the photometric throughput for each mask by doing aperture photometry with our internal source in the Open filter. We used apertures with \SI{175}{\pixel} radius and a \SI{10}{\pixel} wide annulus from \SIrange{200}{210}{\pixel} for background subtraction and calculate the relative flux values (in \si{\electron/\second}). The photometric throughput is consistently lower than the geometric throughput, with a relative ratio of $T_{phot}=0.93\cdot T_{geom}$. The mask specifications and throughput measurements are listed in \autoref{tbl:pupil}.


\begin{figure}
    \centering
    \script{pupil_masks.py}
    \includegraphics{figures/pupil_masks.pdf}
    \caption{Images of the VAMPIRES pupil and coronagraphic pupil masks. (Top left) the VAMPIRES pupil without any masks. (Top right) the reference pupil laser-cut mask. (Bottom left) the laser-cut Lyot stop optimized for throughput. (Bottom right) the gold-plated Lyot stop from \citep{lucas_visible-light_2022}.\label{fig:pupil_masks}}
\end{figure}

\begin{deluxetable}{lcccccc}
\tabletypesize{\scriptsize}
\tablehead{\multirow{2}{*}{Name} & \multirow{2}{*}{Type} & \colhead{$D_\mathrm{in}$} & \colhead{$D_\mathrm{out}$} & \colhead{$w_s$} & \colhead{$T_{geom}$} & \colhead{$T_{phot}$}  \vspace{-0.75em}\\
    & & \colhead{(\si{mm})} & \colhead{(\si{mm})} & \colhead{(\si{\micron})} & \colhead{(\%)} & \colhead{(\%)}
}
\tablecaption{VAMPIRES Lyot stop specifications.\label{tbl:pupil}}
\startdata
Telescope Pupil & & \num{1300} & \num{7950} & & - & -\\
SCExAO Pupil & C & & 14 & & 100 & 100 \\
\tableline
LyotStop-S & M & 2.14 & 7.06 &  & 98.2 & 88.3 \\
LyotStop-M & M & 2.82 & 6.99 &  & 89.4 & 79.1 \\
LyotStop-L & G & 3.16 & 6.33 &  & 66.1 & 61.4 \\
\enddata
\tablecomments{The telescope pupil is the effective pupil at the Nasymth IR platform .The stop type is either ``G'' for the gold-plated glass window, ``M'' for the laser-cut metal sheet foil, or "C" for laser-cut carbon fiber sheets. The geometric throughput is measured using the pupil imaging lens and the internal source.}
\end{deluxetable}

\subsection{Coronagraphic point-spread function}

We show the coronagraphic PSF for all focal plane masks and all Lyot stops in \autoref{fig:bench_coro_profiles} using the Open filter and the SCExAO internal source.

\subsection{Calibration speckles}

VAMPIRES uses the SCExAO deformable mirror to create calibration speckles (``astrogrid'') for precise astrometry and photometry of the star behind the coronagraph mask \citep{sahoo_precision_2020}. The calibration speckles are typically configured to produce an ``X'' with separations of 10.3, 15.5, or 31.0 $\lambda/D$. The relative photometry of the calibration speckles were measured in multiband mode, for each of the above separations, and at different DM probe amplitudes. 

The relative flux is measured by taking the peaks of the convolved calibration speckles divided by the peak of the on-axis PSF. We fit bivariate quadratic functions of probe amplitude and wavelength to the relative PSF photometry (\autoref{eqn:astrogrid}). We note that these functions are only valid for a single astrogrid pattern due to the DM influence function and for DM amplitudes below 1 radian wavefront distortion ($\sim$\SI{95}{\nano\meter}). They also may vary with AO loop speed, but we do not characterize that here. We also fit the coefficients for the ``waffle'' spots passively created by the gridding of the DM. We report our model fits in \autoref{tbl:astrogrid}.
\begin{equation}
    \label{eqn:astrogrid}
    \frac{f_c}{f_*}\left( \lambda, A_{DM} | c \right) = c \cdot A_{DM}^2 / \lambda^2
\end{equation}


TODO READ RESULTSThe power law scaling fits well except for the brightest, closest calibration speckles.

To minimize the negative effects of astrogrid (namely photon noise in the control radius) the amplitude and separation should be tuned so the pattern is barely detectable. There is essentially a trade-off between centroid accuracy due to astrogrid S/N and contrast, and we recommend observers aiming for a speckle brightness a factor of a few above the PSF brightness at that separation. Using \autoref{fig:onsky_psf_profiles} this corresponds to roughly $<10^{-3}$ at 10$\lambda/D$ (\SI{7}{\nano\meter} astrogrid) and $<$\num{5e-4} at 15$\lambda/D$ (\SI{7}{\nano\meter} astrogrid).

There are some practical considerations for using astrogrid, though, especially when using VAMPIRES simultaneously with other modules of SCExAO. For example, when using SCExAO/CHARIS for PDI the Wollaston prism forces the use of the 10.3$\lambda/D$ separation grid so that the speckles all land within the FOV in broadband mode. Similarly, the speckle brightness in the near-infrared is fainter for a given pattern amplitude, which means you may have brighter speckles in VAMPIRES than desired to achieve satisfactory S/N in CHARIS. Lastly, there are the effects of chromaticity for wide filters (like the Open filter) and for the widely separated astrogrid patterns, which may lower the achievable astrometric or photometric precision.

\begin{deluxetable}{lcc}
\tabletypesize{\small}
\tablehead{\colhead{Pattern} & \colhead{Separation ($\lambda/D$)} & \colhead{c}}
\tablecaption{Astrogrid relative photometry flux scaling.\label{tbl:astrogrid}}
\startdata
XYgrid & 10.3 & \num{11.5\pm0.007}\\
XYgrid & 15.5 & \num{5.97\pm0.005}\\
XYgrid & 31.0 & \num{0.490\pm0.001}\\
\enddata
\tablecomments{Photometry measured using \SI{12}{\pixel}-radius apertures. Astrogrid applied using \SI{1}{\kilo\hertz} modulation speed.}
\end{deluxetable}

\begin{figure}
    \centering
    \script{astrogrid_photometry.py}
    \includegraphics{figures/astrogrid_photometry.pdf}
    \caption{Relative photometric flux of the astrogrid calibration speckles. Each plot corresponds to a different speckle pattern and each curve corresponds to a given pattern amplitude (in mechanical microns applied to the DM). \label{fig:astrogrid_photometry}}
\end{figure}


\subsection{On-sky contrast curves}

We have measured the surface brightness profiles and contrast curves for the CLC-3 and CLC-5 coronagraph masks. Data were obtained from observations of HD 102438 and GL 758 (HD 182488). The data were processed using dark subtraction and flat correction. The data were taken in the multiband-imaging mode so for each field the astrogrid calibration speckles were used to register the frame and we combined the coregistered and median-collapsed frames into a spectral cube. We use the aperture photometry of the calibration speckles for flux conversion to contrast using the relative scaling factors (\autoref{eqn:astrogrid},\autoref{tbl:astrogrid}).

The surface brightness profiles are plotted in TODO. We used TODO algorithm for modeling the PSF for each wavelength. The 1$\sigma$ small-sample statistics corrected contrast curves \citep{mawet_fundamental_2014} are plotted in TODO. We also calculate the double-SDI contrast curve after radially scaling each field by its wavelength and subtracting the median frame before rescaling and collapsing.

\subsection{Residual atmospheric dispersion}

There is only a single broadband atmospheric dispersion corrector (ADC) in the common path of VAMPIRES. This can result in residual atmospheric dispersion, depending on the airmass. Residual dispersion can be seen as a shift in the PSF over different wavelengths range. When using a fixed coronagraph, this can cause the stellar PSF to actually appear only partially covered by the coronagraph mask. This is often seen in the K-band of coronagraphic low-dispersion CHARIS observations. With multiband imaging we can see this effect (\autoref{fig:resid_adc}), but it only affected the CLC-3 mask at airmasses above $\sim$\num{2.2} ($<$\ang{27} elevation). The larger masks (CLC-5 and CLC-7) would be less affected by the dispersion, but at a loss of inner working angle.

\begin{figure}
    \centering
    \script{resid_adc.py}
    \includegraphics{figures/resid_adc.pdf}
    \caption{Coronagraphic multiband imaging of HD163296 with the CLC-3 coronagraph at 2.25 airmass (\ang{26.3} elevation). The filter is labeled in the top left, the location of the star is marked with a cross, and the telescope elevation and azimuth axes are displayed with a compass in the bottom left. The image is shown with a square-root stretch with separate limits for each filter.\label{fig:resid_adc}}
\end{figure}

\subsection{Redundant apodizing pupil}

A redundant apodized pupil mask (RAP; PI: Leboulleux, L.) was deployed on VAMPIRES in September 2023. This mask apodizes the pupil in a way which creates a static dark hole that is robust to low-wind effect up to $\sim$\SI{1}{rad} RMS wavefront error. The trade-off is the limited inner and outer working angles and the fact that there is no attenuation of the stellar PSF, which may lead to saturation. We show the RAP mask, PSF, and profile from bench tests in the Open filter in \autoref{fig:rap}. Further testing on-sky is required before opening to observers and is future work.

\begin{figure}
    \centering
    \script{rap.py}
    \includegraphics{figures/rap.pdf}
    \caption{The redundant apodized pupil (RAP) mask. (Top left) a pupil image of the mask showing the apodization pattern. (Top right) a \ang{;;2}x\ang{;;2} FOV of the PSF produced by the RAP using the SCExAO internal source in the Open filter. The image is stretched to emphasize the dark hole from approximately \ang{;;0.1} to \ang{;;0.8}. (Bottom) a normalized radial profile of the top right PSF in log scale.\label{fig:rap}}
\end{figure}