\subsection{Narrowband Imaging of the R Aqr Nebula\label{sec:raqr}}

VAMPIRES high angular resolution, dynamic range, and sensitivity are highlighted by emission line imaging of the innermost regions of the jets and binary of \object{R Aqr}. R Aqr is a symbiotic binary-- the primary component is an AGB star (a Mira variable) and the secondary is a white dwarf (WD) \citep{merrill_partial_1935,merrill_spectra_1940}. Due to the proximity of the WD (\SIrange{10}{50}{\mas} over the orbit), matter is transferred directly from the giant star onto a compact accretion disk around the WD, creating a bipolar jet that extends $\sim$\ang{;;30} to the NE and SW \citep[and references therein]{schmid_spherezimpol_2017}. The \SI{42}{yr} orbit of the binary has been studied in detail using radial velocity measurements, and recently with visual ephemerides from visible and radio imaging \citep{gromadzki_spectroscopic_2009,bujarrabal_high-resolution_2018,alcolea_determining_2023}. 

The WD companion has not been detected directly in the visible/NIR; at radio wavelengths, the CO continuum can be mapped with sufficient angular resolution to resolve the WD \citep{bujarrabal_high-resolution_2018,alcolea_determining_2023}. \citet{schmid_spherezimpol_2017} showed using ZIMPOL that there is a bright H$\alpha$ emission feature coincident with the WD, thought to be a compact accretion disk and the origin of the bipolar jet. In 2020, a periastron passage occurred, putting the WD only a few \si{au} from the AGB star. This periastron passage was studied at many wavelengths \citep{hinkle_2020_2022,sacchi_front-row_2024}.

R Aqr was observed with VAMPIRES on \formatdate{7}{7}{2023} using the narrowband H$\alpha$/H$\alpha$-continuum filter pair. Relatively long exposures were used (\SI{1.2}{s}), which blurred the PSF and reduced the efficacy of lucky imaging (\autoref{fig:lucky_imaging}). The seeing during the observation was \ang{;;0.8}$\pm$\ang{;;0.2} from the MKWC DIMM data, and the FWHM in the continuum image was \SI{32}{\mas} with a Strehl ratio of 15\%. The collapsed H$\alpha$ image is shown in \autoref{fig:raqr}, and a side-by-side of the inner region with the continuum is shown in \autoref{fig:raqr_mosaic}.


\begin{figure}[h]
    \centering
    \script{RAqr.py}
    \includegraphics[width=\columnwidth]{figures/20230707_RAqr_Halpha.pdf}
    \caption{VAMPIRES H$\alpha$ narrowband image of R Aqr with logarithmic stretch showing the emission nebula. A faint diffraction spike, which is not astrophysical, can be seen roughly aligned with the RA axis around the central source.\label{fig:raqr}}
\end{figure}

\begin{figure}[h]
    \centering
    \script{RAqr.py}
    \includegraphics[width=\columnwidth]{figures/20230707_RAqr_mosaic.pdf}
    \caption{A zoom-in of the inner \ang{;;0.24} FOV of R Aqr. (left) H$\alpha$ narrowband image shows the AGB star and a bright jet emanating from the WD. The expected location of the WD from the orbit of \citet{alcolea_determining_2023} is marked with a white circle one FWHM in width. (right) H$\alpha$-continuum image, which only contains emission from the AGB star.\label{fig:raqr_mosaic}}
\end{figure}

The data shows a resolved PSF-like clump to the northeast and diffuse emission extending from NE to SW. A separation of \SI{50.0}{\mas} at a position angle of \ang{65.0} east of north was measured using PSF models. 

Interestingly, this does not match the observations of \citet{bujarrabal_high-resolution_2018,alcolea_determining_2023} (\autoref{fig:raqr_mosaic}), the H$\alpha$ emission clump is no longer co-located with the WD companion. Unpublished high-resolution ALMA CO observations from a month after these observations show that the WD now has an asymmetric jet following its periastron passage (Alcolea,~J., priv. communication). The location of the asymmetric jet is consistent with the observed H$\alpha$ emission north of the actual position of the WD. Future H$\alpha$ observations tracking the evolution of the jet emission are planned and will provide insight into the dynamical evolution of the R Aqr system.
