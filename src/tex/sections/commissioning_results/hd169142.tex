\subsection{Spectro-Polarimetric Imaging of HD 169142\label{sec:hd169142}}

VAMPIRES spectro-polarimetric capabilities are highlighted with multiband imaging of the young star \object{HD 169142}. This system hosts a protoplanetary disk \citep{quanz_gaps_2013} and emission signatures from potential protoplanets \citep{hammond_confirmation_2023}. The disk itself is nearly face-on ($i{\sim}$\ang{13}), and includes two rings ($R_{\rm in}{\sim}$\SI{170}{\mas}, $R_{\rm out}{\sim}$\SI{720}{\mas}), with a gap between them. A cavity is close to the star within the inner ring \citep{fedele_alma_2017}.

This system has been imaged in polarized light by \citet{kuhn_imaging_2001,quanz_gaps_2013,monnier_polarized_2017,pohl_circumstellar_2017}, including in the visible with ZIMPOL by \citet{bertrang_hd_2018}. The polarized images match the radio continuum disk morphology, comprising a bright inner ring with a fainter outer ring. The disk is nearly face-on, with an estimated inclination close to \ang{13} \citep{fedele_alma_2017}.

HD 169142 was observed with VAMPIRES in the ``slow'' polarimetry mode (no FLC) with the \SI{105}{\mas} Lyot coronagraph for \SI{55}{minutes}, resulting in \SI{47}{minutes} of data-- roughly an 85\% observing efficiency, including all overheads from external triggering and HWP rotation. The average seeing for the night from the Maunakea weather center (MKWC) archive\footnote{\url{http://mkwc.ifa.hawaii.edu/current/seeing/index.cgi}} was \SI{0.8\pm0.2}{''}.

The data was reduced as described in \autoref{sec:processing} with background subtraction and double-difference polarimetric reduction. Mueller matrices were adapted from \citet{zhang_characterizing_2023} for the common-path HWP, image rotator, and generic optics term (i.e., excluding the detector intensity ratio and the FLC, since these components were modified in the upgrade). Residual instrumental polarization was corrected using the photometric sums in an annulus from \ang{;;0.24} to \ang{;;0.33}, which matches the cavity of the disk \citep{bertrang_hd_2018}.

The Stokes $Q_\phi$ frames, which are approximately equal to the polarized intensity, are shown in \autoref{fig:hd169142_mosaic}. The inner disk is evident, with some diminished intensity to the southwest. A version scaled by the squared stellocentric distance to normalize the stellar irradiation is also shown, highlighting the disk gap and the inner rim of the outer disk.

\begin{figure*}[t]
    \centering
    \script{HD169142.py}
    \includegraphics[width=0.9\textwidth]{figures/20230707_HD169142_Qphi_mosaic.pdf}
    \caption{VAMPIRES polarimetric observations of HD 169142 in multiband imaging mode. The Stokes $Q_\phi$ frames are cropped to the inner \ang{;;1.2} FOV with the coronagraph mask marked with a black circle. The data from each multiband filter is shown in the left four quadrants, and the wavelength-collapsed data is shown in the right column. The bottom-right image is multiplied by the squared stellocentric distance ($r^2$), which better shows the edge of the ring at $\sim$\ang{;;0.4}. All data are rotated north up and east left and shown in a linear scale with separate limits for all images.\label{fig:hd169142_mosaic}}
\end{figure*}

Radial profiles with bin sizes of 4 pixels from each $Q_\phi$ frame and each Stokes $I$ frame were measured from the coronagraph IWA (\ang{;;0.1} out to \ang{;;1.4}). The ratio of these profiles is shown alongside the $Q_\phi$ profiles in \autoref{fig:hd169142_flux}. The mean integrated partial-polarization of the disk is \SI{0.6}{\%} across all wavelengths and the scattering intensity is higher at redder wavelengths. 

\begin{figure}[h]
    \centering
    \script{HD169142.py}
    \includegraphics[width=\columnwidth]{figures/20230707_HD169142_Qphi_flux.pdf}
    \caption{Radial profiles from multiband polarimetric images of HD 169142 for the polarized intensity (top; taken from $Q_\phi$ frame) and fractional polarized flux (bottom; $Q_\phi/I$). The profiles are taken with annuli of a width of 4 pixels starting from the coronagraph IWA (\si{105}{mas}). The region of the inner disk shows clearly that there is more dust scattering at redder wavelengths.\label{fig:hd169142_flux}}
\end{figure}

Future work includes modeling geometric disk profiles to analyze the scattering geometry of this disk, allowing a measurement of the scattering phase function of the dust at multiple wavelengths. These measurements give color information about the scattering properties of the dust, which generally requires multiple observations for each filter or integral-field spectroscopy, but with VAMPIRES, they are obtained in one observing sequence using multiband imaging.
