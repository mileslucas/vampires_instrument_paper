\section{First Light Results}\label{sec:firstlight}

The VAMPIRES upgrades were tested on-sky from June to August 2023, some results from our first-light observations are presented below.

\subsection{HD 169142\label{sec:hd169142}}

We observed the young star HD 169142, which is a known protoplanetary disk host, on \formatdate{7}{7}{2023}. We used the multiband imaging mode without the FLC (slow polarimetry mode) with the \SI{105}{\mas} Lyot coronagraph for XX hours, resulting in XX minutes of data- roughly a YY\% observing efficiency. We reduced this data as described in \autoref{sec:processing} with background subtraction and double-difference polarimetric reduction with an idealized Mueller-matrix correction. We used an annulus from \ang{;;0.24} to \ang{;;0.33} for removing residual instrumental polarization since this matches the expected region of the cavity of the disk. The $Q_\phi$ frames from each filter as well as the wavelength-collapsed frame are shown in TODO. We note that using an idealized Mueller matrix is not optimal, but future analysis will be done when the polarization characteristics for the upgraded instrument are fit.

\begin{figure*}[t]
    \centering
    \script{HD169142.py}
    \includegraphics[width=\textwidth]{figures/20230707_HD169142_Qphi_mosaic.pdf}
    \caption{\formatdate{7}{7}{2023} VAMPIRES observations of HD 169142 in multiband imaging mode. Each column represents one multiband filter. The top row is the Stokes $Q_\phi$ image in linear scale (different scale for each filter). The bottom row is  Stokes $Q_\phi\times r^2$, where $r$ is the stellocentric distance, to normalize the stellar irradiation (different scale for each filter).\label{fig:hd169142_mosaic}}
\end{figure*}

We take each $Q_\phi$ frame and integrate the flux out to \ang{;;1} and compare to the same flux value measured in the Stokes I frame. The relative fluxes for each filter are shown in TODO. The integrated partial-polarization of the disk roughly matches the expectations given similar measurements from TODO and TODO. These measurements give us color information about the scattering properties of the dust which can only be achieved in broadband filters over multiple observations, but we achieve in a single observing sequence thanks to the multiband imaging technique.

\begin{figure}
    \centering
    \script{HD169142.py}
    \includegraphics[width=\columnwidth]{figures/20230707_HD169142_Qphi_flux.pdf}
    \caption{\formatdate{7}{7}{2023} VAMPIRES HD 169142 Integrated polarized flux (taken from $Q_\phi$ frame) and fractional polarized flux.\label{fig:hd169142_flux}}
\end{figure}

In future work we plan to model and fit geometric disk profiles to analyze the scattering geometry of this disk, allowing a measurement of the scattering phase function of the dust at multiple wavelengths. This will enable estimation of disk scattering properties like the SLDKFJSLDKFJLSDKFJSLDKJFS.

\subsection{R Aqr\label{sec:raqr}}

\subsection{Neptune\label{sec:neptune}}
