\section{First light results}\label{sec:firstlight}

The VAMPIRES upgrades were tested on-sky from June to August 2023, some results from our first-light observations are presented below.

\subsection{HD 169142\label{sec:hd169142}}

We observed the young star \textit{HD 169142}, which is a known protoplanetary disk host, on \formatdate{7}{7}{2023}. We used the multiband imaging mode without the FLC (slow polarimetry mode) with the \SI{105}{\mas} Lyot coronagraph for \SI{55}{minutes}, resulting in \SI{47}{minutes} of data- roughly an \SI{85}{\%} observing efficiency. We reduced this data as described in \autoref{sec:processing} with background subtraction and double-difference polarimetric reduction with an idealized Mueller-matrix correction. We used an annulus from \ang{;;0.24} to \ang{;;0.33} for removing residual instrumental polarization since this matches the expected region of the cavity of the disk \citep{bertrang_hd_2018}. We convert the data from \si{adu} to \si{mJy/arcsec^2} using a synthetic photometry of a standard F2V model spectrum \citep{pickles_stellar_1998} normalized to i=\SI{7.978}{mag} \citep{zacharias_fourth_2013}. We note that using an idealized Mueller matrix is not optimal, but future analysis will be done when the upgraded instrument polarization calibration is completed.

The $Q_\phi$ frames from each filter, which we use as the polarized intensity, are shown in \autoref{fig:hd169142_mosaic}. The inner disk is visible at all wavelengths. We also show a version scaled by the squared stellocentric distance to normalize the stellar irradiation. The scaled version shows the large disk gap and the inner rim of the outer disk. We note that compared to images from \cite{bertrang_hd_2018} we have worse S/N and do not fully resolve the outer disk. However, our observations have smaller spectral bandwidth per image and use a fraction of the detector integration time (\SI{1}{s} versus \SIrange{6}{10}{s}). Future observations with longer detector integration times are needed to resolve the full disk structure.

\begin{figure}
    \centering
    \script{HD169142.py}
    \includegraphics[width=\columnwidth]{figures/20230707_HD169142_Qphi_mosaic.pdf}
    \caption{\formatdate{7}{7}{2023} VAMPIRES observations of \textit{HD 169142} in multiband imaging mode. The data from each filter has been summed together. (left) the Stokes $Q_\phi$ image in linear scale. (right) Stokes $Q_\phi\times r^2$, where $r$ is the distance from the image center in linear scale, which better shows the inner edge of the ring at $\sim$\ang{;;0.4}. All data are rotated so that North is up and East is to the left.\label{fig:hd169142_mosaic}}
\end{figure}

We create radial profiles with bin sizes of 4 pixels from each $Q_\phi$ frame and each Stokes $I$ frame from the coronagraph IWA (\ang{;;0.1} out to \ang{;;1.4}). We take the ratio of these profiles and show them alongside the $Q_\phi$ profiles in \autoref{fig:hd169142_flux}. The mean integrated partial-polarization of the disk is \SI{0.6}{\%} across all wavelengths and the scattering intensity is higher at redder wavelengths. 

\begin{figure}
    \centering
    \script{HD169142.py}
    \includegraphics[width=\columnwidth]{figures/20230707_HD169142_Qphi_flux.pdf}
    \caption{\formatdate{7}{7}{2023} VAMPIRES \textit{HD 169142} radial profiles for the polarized intensity (top; taken from $Q_\phi$ frame) and fractional polarized flux (bottom; $Q_\phi/I$). The profiles are taken with annuli of a width of 4 pixels starting from the coronagraph IWA (\si{105}{mas}). The region of inner disk shows clearly that the dust scatters more at redder wavelengths.\label{fig:hd169142_flux}}
\end{figure}

In future work we plan to model and fit geometric disk profiles to analyze the scattering geometry of this disk, allowing a measurement of the scattering phase function of the dust at multiple wavelengths. These measurements give us color information about the scattering properties of the dust which normally requires multiple observations for each filter, but we obtain them in a one observing sequence by using the multiband imaging technique.

\subsection{R Aqr\label{sec:raqr}}

We performed narrowband H$\alpha$ observations of the Mira variable \textit{R Aqr} on \formatdate{7}{7}{2023} with VAMPIRES. We decided to slightly increase our observing efficiency by not switching the differential filters between each camera- we were not concerned with non-common path aberrations for this short demonstration sequence.

For each camera (and therefore each narrowband filter) we reduced the data following \autoref{sec:processing} with dark subtraction and phase cross-correlation centroiding. We derotate all images to north up east left before coadding with a pixel-by-pixel median. We use calibration factors from \autoref{tbl:filters} and the detector gain and exposure time to convert data from \si{adu} into units of \si{mJy/arcsec^2}. The collapsed images are shown in \autoref{fig:raqr}.

\begin{figure}
    \centering
    \script{RAqr.py}
    \includegraphics[width=\columnwidth]{figures/20230707_RAqr_Halpha.pdf}
    \caption{\formatdate{28}{6}{2023} VAMPIRES \textit{R Aqr} reduced H$\alpha$ filter image with logarithmic stretch showing the emission nebula.\label{fig:raqr}}
\end{figure}

\begin{figure}
    \centering
    \script{RAqr.py}
    \includegraphics[width=\columnwidth]{figures/20230707_RAqr_mosaic.pdf}
    \caption{\formatdate{28}{6}{2023} VAMPIRES \textit{R Aqr} Reduced images in linear stretch. Both images are in units of \si{mJy/arcsec^2} and have the same limits. (left) The H$\alpha$ image showing the giant star and a bright emission clump to the northeast. (right) The H$\alpha$-continuum image which only contains emission from the giant star.\label{fig:raqr_mosaic}}
\end{figure}

\subsection{Neptune\label{sec:neptune}}

We observed gas giant \textit{Neptune} on \formatdate{11}{7}{2023} in the multiband imaging mode to test VAMPIRES capabilities for planetary astromony. \textit{Neptune} was a good target due to its angular diameter being within the VAMPIRES FOV ($<$\ang{;;3}) and without being too faint. The adaptive optics systems during this observation were not able to correct for wavefront errors with the typical performance due to the lack of a point-source guide star- rather using the extended planet itself as a guide star. In addition, this data requires slight modifications in the post-processing to address the unique features of a planetary image.

\begin{deluxetable}{ll}
\tablewidth{\columnwidth}
\tablehead{
    \colhead{Parameter} & 
    \colhead{Value}
}
\tablecaption{Ephemeris of Neptune for our observations on \formatdate{11}{7}{2023} based on JPL horizons data.\label{tbl:neptune}}
\startdata
Apparent diameter & \ang{;;2.313102} \\
North pole angle & \ang{318.0816} \\
North pole distance & \ang{;;-1.058} \\
Solar phase angle & \ang{1.8173}
\enddata
\tablereferences{\url{https://ssd.jpl.nasa.gov/horizons/}}
\end{deluxetable}


For processing, we attempted flat-field correction with some success. Multiband observations are uniquely difficult to flat field, because a single detector image has four images at different wavelengths, and therefore different intensities when illuminated with a calibration laser or flat lamp. We developed a simple computer-vision algorithm to fit rotated rectangles to each flat-field and normalizes each field with a local value. Unfortunately, in this data there was a bug in the data acquisition software which truncated all data to 15-bit unsigned integers, artificially saturating readout values greater than $2^{15}$. This has been fixed, but our calibration flats for the F720 filter from this epoch all suffer from this truncation leading to flat-field errors (as can be seen in \autoref{fig:neptune_mosaic}). We also discovered some imperfect flat-field correction of some dust features in other fields, likely due to small differences in the optical alignment of the glass fieldstop.

We did not attempt to centroid this data, instead we registered each frame using the field positions fit during our flat-field procedure. Future work would benefit from more precise centroiding, perhaps using a radon transform to fit the disk position for each image. We use calibration factors from \autoref{tbl:filters} and the detector gain and exposure time to convert data from \si{adu} into units of \si{Jy/arcsec^2}. Similar to \autoref{sec:hd169142} we perform double-difference polarimetric reduction with an idealized Mueller-matrix correction. \citet{schmid_limb_2006} notes that the disk integrated $Q/I$ and $U/I$ frames for Neptune are effectively zero within systematic errors due to its radially-symmetric limb polarization, so we used an aperture the size of the planet for instrument polarization measurement and correction.

We derotate all Stokes images to north up and east left before combining with a pixel-by-pixel median. We noticed in this data that the Stokes Q and U frames do not perfectly align with the expected \ang{45} and \ang{0} nulls, respectively- especially at shorter wavelengths (F620, F670). We assume this is due to imperfect modeling of the instrumental polarization effects in the Mueller-matrix model, which is left to future work.

We calculate an alternative the radial Stokes parameters, $Q_r$ and $U_r$ \citep{schmid_limb_2006},
\begin{align}
    Q_r &= Q\cos{\left(2\theta\right)} + U\sin{\left(2\theta\right)} \\
    U_r &= -Q\sin{\left(2\theta\right)} + U\cos{\left(2\theta\right)}
\end{align}
For an optically thick planetary atmosphere like Neptune's, the most probable alignment of the electric field is radially from the center of the planet to the limb\citep{schmid_limb_2006}. This makes $Q_r$ approximately equal to the polarized intensity and we expect $U_r$ to be minimized, at least to first order. We attempt to correct residual polarization errors in our data by offsetting the angle, $\theta$, for the radial Stokes parameters in each filter and optimizing until $|\Sigma{U_r}|$ is minimized. The required angle offset for each filter was \ang{21.3}, \ang{5.6}, \ang{0.3}, and \ang{1.0}, respectively.

We show the corrected radial Stokes $Q_r$ frames and Stokes $I$ frames in \autoref{fig:neptune_mosaic}. We see clear limb polarization in all filters with little polarization in the center of the image and the maximum polarization at the edge. We see roughly equal polarization along the whole limb (as opposed to concentration at the poles or equator). The total intensity frames show some signs of a polar cloud along with some mild banding structures, especially in the F720 frame (disregarding any flat-field errors). We plot radial profiles of the $Q_r$, $I$, and $Q_r/I$ for each filter in \autoref{fig:neptune_flux}.

\begin{figure*}[t]
    \centering
    \script{neptune.py}
    \includegraphics[width=\textwidth]{figures/20230711_Neptune_mosaic.pdf}
    \caption{\formatdate{11}{7}{2023} VAMPIRES observations of \textit{Neptune} in multiband imaging mode. Each column represents one multiband filter. All data are rotated so that north is up and east is to the left. The top row is the Stokes $Q_r$ image in linear scale (different scale for each filter). The bottom row is Stokes $I$ (different scale for each filter). The apparent diameter of the disk is shown with a circle and the southern polar axis is designated with a line (based on JPL horizons ephemerides). Note there are spurious features in the total intensity images due to flat-field errors (dust, etc.). \label{fig:neptune_mosaic}}
\end{figure*}

\begin{figure}[t]
    \centering
    \script{neptune.py}
    \includegraphics[width=\columnwidth]{figures/20230711_Neptune_flux.pdf}
    \caption{Radial profiles of polarimetric observations of Neptune. All profiles use apertures 10 pixels wide. (top) total intensity. (middle) radial stokes intensity. (bottom) the flux-weighted, \label{fig:neptune_flux}}
\end{figure}
